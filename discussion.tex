\section{Discussion}

Deflection Theory is a novel physical argument corroborated by experimental data and improves general understanding of lift and drag.
It is important to remember that for a complete performance prediction model, Deflection Theory requires another method to compute \(C_N\).
In this paper, Deflection Theory was used to enhance Thin Airfoil Theory.
The same can be easily done for Vortex Panel Methods.

Deflection Theory explains drag produced by unsteady means, including when no net lift is generated.
Freestream deflection oscillation such that the mean lift is zero will result in a non-zero drag.
Similarly, simultaneous steady opposing flow deflections can also produce non-zero drag without net lift.

It was expected for the model derived in this paper to match well with experiment until stall, because Thin Airfoil Theory was used to compute \(C_N\).
A stalled airfoil is not an effective deflector of the freestream flow, but the Deflection Theory still holds.
Computing the \(C_N\) is not the responsibility of Deflection Theory.

Currently the lack of a method to determine the extent of the freestream flow that interacts with the airfoil, represented by the parameter \(A_\infty / A_w\), is the most glaring deficiency.
However, it might be reasonable to empirically determine an estimate for a set of similar airfoils.


%Citation example \cite{Figueredo:2009dg}.

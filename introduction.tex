\section{Introduction}
Current airfoil performance models for incompressible and inviscid flow predict lift without drag.
Conservation of energy dictates that airfoil forces cannot be parallel to freestream flow at the airfoil,
and this is the reason for d'Alembert's Paradox.
Aerodynamic forces in inviscid and incompressible potential flow are therefore normal to the flow direction at the airfoil
, and drag comes from this change in flow direction.
In the three-dimensional context, trailing wake vortex modeling tilts flow at the airfoil to cause "lift-induced drag".
However, no well-known direct relationship between lift and drag exists for the purely two-dimensional context.
Deflection Theory offers an explanation for two-dimensional "lift-coupled drag".

Thin Airfoil Theory and the Vortex Panel Method assume a fixed freestream flow, independent of any forces exerted on the airfoil.
By the conservation of momentum, freestream flow downstream of the airfoil should be deflected to oppose the resultant aerodynamic forces on the airfoil.
Dragless lift is a valid transient solution, but not a steady-state one.
Deflection Theory is able to predict drag as naturally as lift because the conservation of momentum is applied to the portion of freestream flow that interacts with the airfoil.

The common lift-producing spinning cylinder example clearly shows the deficiency of current lift and drag analysis methods.
Everyone would agree that generally speaking, an airfoil produces lift by "pushing down" air, and this could be said of any object that produces lift.
However, the traditional spinning cylinder solution has a flow field with velocity magnitudes and streamlines exactly symmetrical from fore and aft of the cylinder.
This clearly implies that just as much air is pulled up as it is pushed down. This clearly violates Newton's Third Law.

